\chapter{How To Use This Template}

Document-preparing WYSIWYG applications like Microsoft Word are general typing tools that are not friendly to the needs of scientific writing. It is difficult to create scientific writing components, such as equations, figures, and citations using such generic applications. Although some tools and plugins can be used with those applications there is an aesthetic issue that can only be mitigated using a typesetting tool, such as \LaTeX. Besides, indexing and referencing are inherently easy to produce which helps the authors to concentrate more on writing than formatting.

Concretely, there are a few good reasons why \LaTeX~should be used for academic writing. 
\begin{enumerate}
    \item \LaTeX~allows you to typeset your thesis/project report beautifully. It helps to structure the document, equations, citations, indexing, etc, and produce high-quality output. 
    \item Editing a \LaTeX~document is easy. A part of the document can be easily moved and rested in the document allowing greater flexibility with the document without disturbing the typesetting. 
    \item Document reformatting using \LaTeX~is super easy. Changing fonts, margins, citation styles, and rearranging can be done in almost no time. 
    \item The scientific publishing industry has built a stable infrastructure around \LaTeX. 
    \item \LaTeX~is free. Besides, the strong presence of a large \LaTeX~community in WWW to help you in producing high-quality documents. 
\end{enumerate}

If you are new to \LaTeX, you are highly encouraged to read and practice the following instructions carefully before commencing your thesis/project report writing. 

\section{Scenario 1: A Student Doing A Thesis}
\label{sec:case1}
A student of the Department of Computer Science and Engineering at the Green University of Bangladesh named \textit{Afsin Fairuz}, Student ID 243014007, is working alone to write her thesis under the supervision of Associate Professor Dr. Md. Mostafizur Rahman. The title of her thesis is ``Blockchain-Based Food Distribution on the Planet Mars.'' \textit{Afsin} has already conducted her research and got her data ready for writing her thesis. She plans to defend the thesis in the Fall of 2028. The Department of Computer Science and Engineering has declared that \textit{Afsin} will defend her thesis in \textbf{Board 3} consisting of Prof. Dr. Md. Saiful Azad (Professor, CSE) as chair, Mr. Shadman Wadith (Lecturer, CSE) as Member, and Mr. Shamsur Rahman (Tech Lead, Tiger IT Ltd) as External Member. The department also declared that the thesis defense date is February 02, 2029. 

\begin{figure}[ht]
    \centering
    \tikz{
      \node[anchor=south west] (image) at (0,0) {\includegraphics[width=0.3\textwidth]{Images/fig0.png}};
      \draw[GUBRed, line width=1mm] (1.6,1) ellipse (30pt and 10pt);
    }
    \caption{Gaining access to the \textit{main.tex} file from the left side panel of Overleaf for doing initial configurations of this template.}
    \label{fig:main}
\end{figure}

Considering the scenario above, \textit{Afsin} needs to do the initial configuration of this template. To do so, she has to click on the \textit{main.tex} file as shown in Figure \ref{fig:main}. Once \textit{main.tex} is open, she should do the initial configurations as shown in the following two subsections.

\subsection{Setting Thesis Particulars}

Since \textit{Afsin} is doing a thesis, she should declare it at the beginning. Thus in the template, she uses the \lstinline[language=latex]!\def\ReportType{Thesis\xspace}! primitive and she comments out the other two \lstinline[language=latex]!\def\ReportType{...\xspace}! primitives to make sure the type of her report is only Thesis. Moreover, the following primitives are to be filled according to the requirements. 
\begin{itemize}
    \item \lstinline[language=latex]!\def\ReportTitle{...\xspace}!,
    \item \lstinline[language=latex]!\def\Supervisor{...\xspace}!, 
    \item \lstinline[language=latex]!\def\SupervisorPosition{...\xspace}!,
    \item \lstinline[language=latex]!\def\reportSubmissionDate{...}!, and
    \item \lstinline[language=latex]!\def\reportSubmissionTerm{...}!.
\end{itemize}

After setting up the thesis particulars, the relevant part of the \textit{main.tex} would look as follows.
\lstset{language=latex,morekeywords={begin},caption={}}
\begin{lstlisting}
%-----------------------------------------------------------
% Set your report/thesis particulars 
%-----------------------------------------------------------
%
% Use one: Thesis/Project report
%\def\RoportType{Project report\xspace} 
\def\RoportType{Thesis\xspace}
%
\def\ReportTitle{Blockchain-Based Food Distribution in the
Planet Mars\xspace}
%
\def\Supervisor{Dr. Md. Mostafizur Rahman\xspace} 
\def\SupervisorPosition{Associate Professor\xspace}
%
%\def\reportSubmissionDate{\today}
\def\reportSubmissionDate{February 02, 2029}
\def\reportSubmissionTerm{Fall 2028}
\end{lstlisting}

\subsection{Setting Author's Particulars}
Based on the description provided in Subsection \ref{sec:case1}, there are no other members/authors in the group except \textit{Afsin}. That means she is the first author of her Thesis. \textit{Afsin} loves her parents, so she dedicates her thesis to her Mother \textit{Mansura Akhter} and Father \textit{Manzurul Haque}. She wants to write the following message on the ``Dedication" page of the thesis.
\begin{center}
\parbox{5cm}{
To my loving mother\\ \textit{
Mansura Akhter} \\ and father \\ \textit{Manzurul Haque}
}   
\end{center}
So, \textit{Afsin} would configure the following primitives as the requirements.
\begin{itemize}
    \item \lstinline[language=latex]!\def\numberOfAuthors{ }!
    \item \lstinline[language=latex]!\def\firstAuthor{...\xspace}!
    \item \lstinline[language=latex]!\def\firstAuthorID{...\xspace}!
    \item \lstinline[language=latex]!\def\firstAuthorDedication{...}!
\end{itemize}

After inserting the author's particulars, the group members' particular part of the \textit{main.tex} would look as follows.
\lstset{language=latex, caption={}}
\begin{lstlisting}
%-----------------------------------------------------------
% Set your group members' particulars 
%-----------------------------------------------------------
%
\def\numberOfAuthors{1} % write 1, 2, or 3 (depends on group)
%
\def\firstAuthor{Afsin Fairuz\xspace} 
\def\firstAuthorID{243014007\xspace}
\def\firstAuthorDedication{To my loving mother\\ \textit{
Mansura Akhter} \\ and father \\ \textit{Manzurul Haque}}
\end{lstlisting}

Please note that \textit{Afsin} does not have to fill up the remaining authors' information although there is a provision to input more authors' information. She can leave them as it will not create any problem in rendering her document.

Next, \textit{Afsin} has to fill in the information regarding the defense board and its member information. Based on the information given, \textit{Afsin} has to edit the following piece of code accordingly.

\lstset{language=latex, caption={}}
\begin{lstlisting}
%-----------------------------------------------------------
% Set Board members' particulars 
%-----------------------------------------------------------
%
\def\BoardNumber{3} % board to defende
%
\def\BoardChair{Prof. Dr. Md Saiful Azad\xspace} 
\def\BoardChairPosition{Professor, Dept. of CSE\xspace}
%
\def\BoardMember{Mr. Shadman Wadith\xspace} 
\def\BoardMemberPosition{Lecturer, Dept. of CSE\xspace} 
%
\def\BoardExternalMember{Mr. Shamsur Rahman\xspace} 
\def\BoardExternalMemberPosition{Tech Lead\xspace} 
\def\BoardExternalMemberAffiliation{Tiger IT Ltd\xspace}
%
%-----------------------------------------------------------
\end{lstlisting}

Now \textit{Afsin} is ready to recompile the document. In order to do that, she finds the \includegraphics[width=0.12\textwidth]{Images/fig1.png} button of Overleaf and click on it. Once recompilation is done, she finds that all the necessary changes are made automatically in the following pages.
\begin{enumerate}
    \item Front page
    \item Copyright page
    \item Declaration page
    \item Endorsement Certificate page
    \item Certificate page
    \item Acknowledgments page
    \item Dedication page, and
    \item Cover page
\end{enumerate}


If you are doing a thesis alone, follow \textit{Afsin} to configure the template according to the requirement. After configuration, you will be able to see the dynamically produced skeleton of your thesis too. Try it yourself! If you have done it successfully, Congratulations to you!

\section{Scenario 2: A Group Doing A Project}
\label{sec:case2}
Three students of the Department of Computer Science and Engineering at the Green University of Bangladesh formed a group to do a  Project under the supervision of Lecturer \textit{Jargis Ahmed}. Names of the students are \textit{Safwan Sadid}, \textit{Aiyan Faiyaz}, and \textit{Aleena Ramin}, and their Student IDs are 243014008, 243014027 and 243014033 respectively. The group has decided the title of their project is ``Design and Development of a Robot for Tree Plantation on Planet Mars.'' The group members have already designed and implemented their project, and now they are ready to write the project report. Upon approval of their supervisor, they are planning to submit the report by the end of Fall 2028. The Department of Computer Science and Engineering has declared that the group will defend her project in \textbf{Board 3} consisting of Prof. Dr. Md. Saiful Azad (Professor, CSE) as chair, Mr. Shadman Wadith (Lecturer, CSE) as Member, and Mr. Shamsur Rahman (Tech Lead, Tiger IT Ltd) as External Member. The department also declared that the thesis defense date is February 02, 2029.  

The group starts writing their project report doing the initial configuration of this template. To do so, first, they click on the \textit{main.tex} file as shown in Figure \ref{fig:main}. Once \textit{main.tex} is open, they do the initial configurations of the template as shown in the following two subsections. 

\subsection{Setting Report Particulars}

Since the group is doing a  project, they use the \lstinline[language=latex]!\def\ReportType{!
\lstinline[language=latex]!Project\xspace}! primitive to make sure the type of their report is a  project report. Further, the following primitives are filled accordingly. 
\begin{itemize}
    \item \lstinline[language=latex]!\def\ReportTitle{...\xspace}!,
    \item \lstinline[language=latex]!\def\Supervisor{...\xspace}!, 
    \item \lstinline[language=latex]!\def\SupervisorPosition{...\xspace}!,
    \item \lstinline[language=latex]!\def\reportSubmissionDate{...}!, and
    \item \lstinline[language=latex]!\def\reportSubmissionTerm{...}!.
\end{itemize}

After configuring the project report particulars, the relevant part of the \textit{main.tex} would look as follows.
\lstset{language=latex,morekeywords={begin},caption={}}
\begin{lstlisting}
%-----------------------------------------------------------
% Set your report/thesis particulars 
%-----------------------------------------------------------
%
% Use one: Thesis/Project report
\def\RoportType{Project report\xspace} 
%\def\RoportType{Thesis\xspace}
%
\def\ReportTitle{Design and Development of a Robot for Tree 
Plantation on Planet Mars\xspace}
%
\def\Supervisor{Jargis Ahmed\xspace} 
\def\SupervisorPosition{Lecturer\xspace}
%
%\def\reportSubmissionDate{\today}
\def\reportSubmissionDate{February 02, 2029}
\def\reportSubmissionTerm{Fall 2028}
\end{lstlisting}

\subsection{Setting Authors' Particulars}
As described in Subsection \ref{sec:case2}, the group consists of three students who are working towards the successful completion of a  project. So, all of them are considered as authors. Based on the contributions made to the project, they came up with an order in the authors' list.

Safwan, Aiyan, and Aleena – all of them decide to dedicate their project report to their favorite persons. Safwan wants to dedicate the work to his parents. The dedication message he wants to write is as follows. 
\begin{center}
\parbox{5cm}{
To my beloved mother\\ \textit{
Munjifa Hasan} \\ and father \\ \textit{M Shohorab Hossain}
}   
\end{center}
 
 Aiyan, on the other hand, loves his grandmother a lot so he decides to dedicate his work to his grandmother. The message goes as follows.
 \begin{center}
\parbox{5cm}{
To my grandmother\\ \textit{
Mahmuda Begum}
}   
\end{center}
 
 Aleena has been influenced a lot by her mentor and her sister. So, she decides to dedicate the work to them. The messages she wants to include in the report are as follows.
\begin{center}
\parbox{5cm}{
I would like to dedicate my work to my mentor\\ \textit{M. Tafazzal Hossain} \\ and my sister \\ \textit{Samara Raniyah}
}   
\end{center}
Accordingly, the team configures the following primitives as the requirements.
\begin{itemize}
    \item \lstinline[language=latex]!\def\numberOfAuthors{ }!
    \item \lstinline[language=latex]!\def\firstAuthor{...\xspace}!
    \item \lstinline[language=latex]!\def\firstAuthorID{...\xspace}!
    \item \lstinline[language=latex]!\def\firstAuthorDedication{...}!
    \item \lstinline[language=latex]!\def\secondAuthor{...\xspace}!
    \item \lstinline[language=latex]!\def\secondAuthorID{...\xspace}!
    \item \lstinline[language=latex]!\def\secondAuthorDedication{...}!
    \item \lstinline[language=latex]!\def\thirdAuthor{...\xspace}!
    \item \lstinline[language=latex]!\def\thirdAuthorID{...\xspace}!
    \item \lstinline[language=latex]!\def\thirdAuthorDedication{...}!
\end{itemize}

After inserting the authors' particulars, the group members' particular part of the \textit{main.tex} would look as follows.
\lstset{language=latex, caption={}}
\begin{lstlisting}
%-----------------------------------------------------------
% Set your group members particulars 
%-----------------------------------------------------------
%
\def\numberOfAuthors{3} % write 1, 2 or 3 (depends on group)
%
\def\firstAuthor{Safwan Sadid\xspace} 
\def\firstAuthorID{243014008\xspace}
\def\firstAuthorDedication{To my beloved mother\\ \textit{
Munjifa Hasan} \\ and father \\ \textit{M Shohorab Hossain}}
%
\def\secondAuthor{Aiyan Faiyaz\xspace} 
\def\secondAuthorID{243014027\xspace} 
\def\secondAuthorDedication{To my grandmother\\
\textit{Mahmuda Begum}}
%
\def\thirdAuthor{Aleena Ramin\xspace} 
\def\thirdAuthorID{243014033\xspace} 
\def\thirdAuthorDedication{I would like to dedicate my work to 
my mentor\\ \textit{M. Tafazzal Hossain} \\ and my sister \\ 
\textit{Samara Raniyah}}
\end{lstlisting}

In the end, Board Members' particular part of the \textit{main.tex} has to be filled up which would look as follows.
\lstset{language=latex, caption={}}
\begin{lstlisting}
%-----------------------------------------------------------
% Set Board members' particulars 
%-----------------------------------------------------------
%
\def\BoardNumber{3} % board to defende
%
\def\BoardChair{Prof. Dr. Md Saiful Azad\xspace} 
\def\BoardChairPosition{Professor, Dept. of CSE\xspace}
%
\def\BoardMember{Mr. Shadman Wadith\xspace} 
\def\BoardMemberPosition{Lecturer, Dept. of CSE\xspace} 
%
\def\BoardExternalMember{Mr. Shamsur Rahman\xspace} 
\def\BoardExternalMemberPosition{Tech Lead\xspace} 
\def\BoardExternalMemberAffiliation{Tiger IT Ltd\xspace}
%
%-----------------------------------------------------------
\end{lstlisting}

Now the team is ready to recompile the document. To do that, they click on the  \includegraphics[width=0.12\textwidth]{Images/fig1.png} button to render the document. Once recompilation is done, they found the necessary changes according to the above configurations on the following pages.
\begin{enumerate}
    \item Frontpage
    \item Copyright page
    \item Declaration page
    \item Endorsement Page
    \item Certificate page
    \item Acknowledgments page
    \item Dedication page, and
    \item Cover page
\end{enumerate}

If your team consists of three members and the team is doing a  project then follow the above steps to complete the initial configurations. If you have done it successfully, to all three of you - Kudos!

\begin{figure}[ht]
    \centering
    \tikz{
      \node[anchor=south west] (image) at (0,0) {\includegraphics[width=0.9\textwidth]{Images/fig6.png}};
      \draw[GUBRed, line width=1mm] (1.9,1.9) ellipse (35pt and 10pt);
    }
    \caption{Write an Abstract of your work in the \textit{abstract.tex} file accessible from the left side panel of Overleaf.}
    \label{fig:abstract}
\end{figure}

\section{Abstract and Keywords}
An abstract is a concise summary of an experiment or project written in usually a paragraph of roughly 250 words. An abstract should contain the information in the given order.
\begin{itemize}
    \item the purpose of the study/project motivation; 
    \item the basic design of the study/solving methods; 
    \item major findings/challenges faced in implementing the project; and, 
    \item a brief summary and conclusions.
\end{itemize} 
In your report, you should write an abstract in the \textit{abstract.tex} file. The location of the \textit{abstract.tex} file is illustrated in Figure \ref{fig:abstract}.

In the same file, there is a provision to include a few relevant keywords for indexing purposes.


\section{Adding a New Chapter}
Now, the template is ready for you to add chapters containing the text that you want to put in your report. This template already contains five chapters which are stored in the \textbf{Chapters} folder. You are free to edit all the chapters according to your needs. Additionally, if you need more chapters to cater to your needs, first, follow the instructions given in Figure \ref{fig:new_chapter}. Then, from the \textbf{Add Files} dialogue box give a file name with \textit{.tex} extension. Finally, click on the \textbf{create} button to confirm your intention. 

\begin{figure}[ht]
    \centering
    \tikz{
      \node[anchor=south west] (image) at (0,0) {\includegraphics[width=0.9\textwidth]{Images/fig2.png}};
      \draw[GUBRed, line width=1mm] (2.2,2.7) ellipse (30pt and 10pt);
      \draw[GUBPurple, line width=1mm,-latex'] (8,6) -- (4.7,6) node [fill=black!5,above,pos=0.4,]{1. Click here};
      \draw[GUBPurple, line width=1mm,-latex'] (8,2.8) -- (3.2,2.8) node [fill=black!5,above,pos=0.23,]{2. Then click here};
    }
    \caption{Create a new chapter from the left side panel of Overleaf of this template.}
    \label{fig:new_chapter}
\end{figure}

\subsection{Customizing Chapter Title}
The newly created chapter is to be prepared for use next. To customize the chapter title, you need to use \lstinline[language=latex]!\chapter{}! command. For example, the newly created chapter title is \textit{Literature Review}. In such case, you should use the following command at the beginning of the newly created chapter file.

\lstset{language=latex, caption={}}
\begin{lstlisting}
\chapter{Literature Review}
\end{lstlisting} 

The above command adds the newly created chapter. Now you are ready to add any content in the the chapter as required. 
\subsection{Including a Chapter in Your Report}

Upon completing the chapter file creation and customizing the title, you can call it from \textit{main.tex}. The primitive to be used in that case is \lstinline[language=latex]!\include{./Chapters/...}!. Please replace $\ldots$ with the chapter name that you created just a while ago.

\lstset{language=latex, caption={}}
\begin{lstlisting}
%-----------------------------------------------------------
% Include Chapters 
%-----------------------------------------------------------
 \setlength{\parskip}{1em}    % Adds space between paragraphs
\setlength{\parindent}{0pt}  % Removes default indentation
\chapter{Introduction}

\section{Background}

Traffic congestion is a persistent and growing problem in urban areas worldwide. As cities expand and vehicle ownership increases, the existing road infrastructure becomes insufficient to accommodate rising traffic volumes. This leads to longer travel times, higher fuel consumption, and increased emissions, which collectively affect both the economy and the environment.

Traditional traffic management systems rely on fixed-timer traffic signals or pre-defined schedules that do not adapt well to dynamic traffic patterns. In many cities, traffic flow is managed manually or through outdated systems that lack the ability to respond in real time to congestion or unusual traffic conditions. These limitations result in inefficient traffic distribution and contribute significantly to gridlock, especially during peak hours or special events \cite{zhao2017improving}.

Recent advancements in artificial intelligence (AI), computer vision, and real-time data processing have created opportunities for more intelligent and responsive traffic control mechanisms. AI-powered systems can leverage live camera feeds, weather updates, historical trends, and other contextual data to dynamically optimize traffic signal timings, helping to alleviate congestion and improve overall flow \cite{wang2020real, chen2018ai}.

This project builds upon these advancements by proposing an AI-Based Traffic Management System that uses object detection models to monitor vehicle density and direction at intersections. Combined with contextual data and historical analysis, the system aims to provide adaptive signal control and a responsive interface for administrators to intervene when necessary. The goal is to create a smarter, scalable traffic solution suited for modern urban infrastructure and future smart city integration.

\section{Problem Statement}

As urban areas continue to grow, traffic congestion has become a major challenge for cities around the world. Traditional traffic management systems rely on fixed signal timings and manual controls, which are often inefficient in handling real-time traffic conditions. These outdated systems lead to increased travel time, fuel consumption, road accidents, and environmental pollution \cite{jia2019traffic}. 

There is a lack of intelligent, responsive systems that can adapt to the constantly changing traffic flow and prioritize emergency situations. Without the use of real-time data and smart decision-making, traffic control remains slow, ineffective, and frustrating for daily commuters. This project aims to address these issues by developing an AI-Based Traffic Management System that can intelligently monitor, analyze, and manage traffic flow in real time to improve overall road efficiency and safety \cite{chen2018ai}.

\section{Purpose of the Project/Thesis}

The purpose of this project is to design and implement an AI-Based Traffic Management System that uses real-time vehicle detection and contextual data to optimize traffic signal control. By leveraging technologies like YOLOv8 \cite{yolov8}, Kotlin, and machine learning, the system aims to reduce congestion, improve traffic flow, and contribute to the development of smart and sustainable urban infrastructure.

\section{Introduction Chapter Example}

With the increasing number of vehicles on the road, managing traffic efficiently has become a serious challenge, especially in busy urban areas. Traditional traffic systems often work on fixed timings and cannot adjust to real-time road conditions, which leads to unnecessary delays, congestion, and frustration among drivers.

To solve this problem, our project introduces an AI-Based Traffic Management System that uses artificial intelligence to monitor and control traffic in a smarter way. The system collects real-time data from various sources like sensors, cameras, and GPS, and uses AI algorithms to analyze traffic flow. Based on the current situation, it adjusts traffic signals and helps manage congestion more effectively. It can also give priority to emergency vehicles and improve overall road safety.

This system aims to reduce traffic jams, save time, lower fuel consumption, and support the development of smart cities by making traffic management more intelligent and responsive \cite{zhao2017improving}.

\section{Historical Observations}

Historically, urban traffic management has relied on fixed-time traffic signals and manual control mechanisms. These systems were designed based on average traffic conditions and follow pre-programmed cycles that do not account for real-time traffic fluctuations. While such methods were initially effective during periods of low vehicle density, they have proven inadequate in handling the dynamic and high-volume traffic observed in modern cities \cite{wang2020real}.

In Bangladesh, traffic congestion has steadily worsened over the past two decades due to rapid urbanization, insufficient infrastructure development, and poor enforcement of traffic regulations. Intersections such as Mirpur 10, Farmgate, and Motijheel in Dhaka experience severe congestion throughout the day, particularly during peak hours. The lack of adaptive signal systems means that vehicles from less congested directions often receive equal or more signal time than heavily congested ones, resulting in inefficient traffic flow and bottlenecks.

Attempts have been made to introduce semi-automatic signal systems, but these are still largely dependent on human monitoring and do not scale well in complex environments. In recent years, some cities have begun exploring smart traffic solutions using sensors and IoT technologies, but these implementations remain limited in scope and have not been fully integrated into the national infrastructure.

These historical trends underline the urgent need for an intelligent, automated, and scalable traffic management system. With the emergence of AI and computer vision, there is now a feasible opportunity to transition from rule-based traffic control to data-driven decision-making systems that can respond in real time to ever-changing road conditions \cite{chen2018ai}. This project is developed in response to these historical shortcomings and aims to bridge the gap between traditional systems and modern urban mobility demands.
 \chapter{Literature Review}

Effective traffic management has become a growing challenge in modern urban areas due to the exponential increase in the number of vehicles and limited road infrastructure. Traditional traffic control systems, which operate based on pre-programmed signal timings, often fail to adapt to dynamic and unpredictable traffic conditions. This results in prolonged waiting times, fuel wastage, and increased levels of air pollution \cite{zhao2017improving}. To overcome these limitations, researchers have increasingly turned to Artificial Intelligence (AI) technologies as a means to enhance the efficiency and responsiveness of traffic management systems.

\section{Review of Existing Literature}

Numerous studies in recent years have demonstrated the potential of AI in optimizing traffic flow. One of the most widely explored techniques is \textbf{machine learning}, particularly \textbf{reinforcement learning}, which enables traffic signal controllers to learn optimal timing policies based on real-time data. These algorithms are capable of minimizing average waiting times and vehicle congestion by continuously adapting to changing traffic patterns. For instance, research by Wang et al. has shown that Deep Q-Network (DQN)-based approaches can outperform traditional fixed-cycle signals by dynamically adjusting green light durations \cite{wang2020real}.

Another significant area of research is \textbf{computer vision}, where techniques such as \textbf{object detection} and \textbf{image classification} are applied to live video feeds from CCTV cameras. Using Convolutional Neural Networks (CNNs), systems can detect and count vehicles at intersections, enabling real-time estimation of traffic density. This method eliminates the need for expensive infrastructure like inductive loop detectors and provides a scalable solution for urban monitoring \cite{chen2018ai}.

In addition, the integration of \textbf{Internet of Things (IoT)} devices has allowed traffic management systems to gather diverse types of data, including vehicle speed, density, and road occupancy. These IoT-based systems use sensors and GPS-enabled units to feed data into AI models, which then make informed decisions about signal timing, congestion prediction, and emergency vehicle prioritization. Several smart cities have started implementing such systems on a trial basis with promising results \cite{jia2019traffic}.

Projects like \textit{Surtrac}, developed by Carnegie Mellon University, and \textit{SCATS} (Sydney Coordinated Adaptive Traffic System) have demonstrated the real-world applicability of AI in traffic control. These systems utilize decentralized decision-making and real-time data analysis to significantly reduce travel time and vehicle idling \cite{zhao2017improving}. However, most of these systems are implemented in technologically advanced urban environments with high infrastructure budgets.

\section{Knowledge Gaps and Research Motivation}

Despite significant progress, many existing AI-based traffic systems are still limited in scope and implementation. Key challenges include the need for high-quality, real-time data, integration with legacy infrastructure, and the computational complexity of AI models. Moreover, many systems are designed for high-income urban areas and do not account for traffic conditions in developing countries, where infrastructure and resource constraints are more pronounced.

This research aims to address these gaps by designing and developing a real-time AI-based traffic management system that uses computer vision for vehicle detection and a machine learning model to dynamically adjust traffic signals. The system is intended to be low-cost, scalable, and suitable for deployment in congested intersections of developing cities. By building upon the foundations laid in previous research and addressing the identified limitations, this study contributes a practical, efficient, and adaptable solution for modern traffic management challenges \cite{yolov8}.
 \chapter{Methodology}

This chapter presents the systematic approach adopted to design, develop, and implement an AI-Based Traffic Management System that monitors and controls vehicle flow at urban intersections. The methodology includes the system's layered architecture, data acquisition, vehicle detection using computer vision, signal control logic, backend communication, and deployment strategy.

\section{System Architecture Overview}

The proposed system follows a layered architecture (see Figure~\ref{fig:architecture}):

\begin{enumerate}
    \item \textbf{Frontend Layer} — A Kotlin Jetpack Compose desktop interface for real-time monitoring, analytics, and control.
    \item \textbf{Backend Layer} — Developed with Ktor, serving REST APIs and WebSocket channels for real-time communication.
    \item \textbf{AI/ML Processing Layer} — A Python module powered by YOLOv8, deployed via ONNX Runtime for efficient object detection.
    \item \textbf{Infrastructure Layer} — Hosted on a Google Cloud Platform (GCP) virtual machine instance.
\end{enumerate}

Each layer communicates using well-defined APIs and messaging protocols, allowing modular testing and independent deployment.

\begin{figure}[H]
\centering
\includegraphics[width=0.95\textwidth]{Images/architecture_diagram.png}
\caption{System Architecture Diagram}
\label{fig:architecture}
\end{figure}

\section{Vehicle Detection Using Computer Vision}

The core of real-time traffic analysis is vehicle detection, performed using a deep learning model based on YOLOv8 \cite{yolov8}. The detection workflow is as follows:

\begin{itemize}
    \item \textbf{Model:} YOLOv8, trained on COCO dataset and fine-tuned where necessary.
    \item \textbf{Classes Detected:} \textit{car}, \textit{motorcycle}, \textit{bus}, and \textit{truck}.
    \item \textbf{Inference:} Executed using ONNX Runtime in Python for optimized performance.
    \item \textbf{Pipeline:}
    \begin{enumerate}
        \item Frames are captured from live CCTV footage or offline video.
        \item Each frame is resized and normalized before inference.
        \item Detections are parsed into bounding boxes and classified by type and direction.
        \item Aggregated vehicle counts per direction are serialized into JSON.
    \end{enumerate}
\end{itemize}

This data is transmitted to the backend using either HTTP POST or persistent WebSocket.

\section{Traffic Signal Decision Logic}

Upon receiving vehicle counts, the backend applies a decision-making algorithm to determine the next green phase.

\subsection{Rule-Based Scheduling Algorithm}

The first approach uses a weighted scoring formula:

\[
\text{Score}_d = \alpha \cdot \text{VehicleCount}_d + \beta \cdot \text{WaitTime}_d
\]

Where:
\begin{itemize}
    \item $d$ represents each direction: North, South, East, or West.
    \item $\alpha$ and $\beta$ are hyperparameters adjusted via simulation.
    \item $\text{WaitTime}_d$ is updated every cycle.
\end{itemize}

The direction with the highest score is granted the green light.

\subsection{Machine Learning-Based Classifier (Optional)}

To improve decision-making under varying traffic conditions, a Decision Tree classifier is trained on historical and synthetic data. Key features include:

\begin{itemize}
    \item Real-time vehicle counts from all directions
    \item Current signal phase and elapsed duration
    \item Time-of-day classification (e.g., peak, off-peak)
    \item Emergency vehicle presence (boolean flag)
\end{itemize}

This model predicts the optimal direction to prioritize for the next cycle \cite{wang2020real}.

\section{Backend Communication and Coordination}

Ktor acts as the central controller for managing system state and interactions:

\begin{itemize}
    \item \textbf{Data Ingestion:} Accepts POST requests from the AI module.
    \item \textbf{Signal Logic Execution:} Applies rule-based or model-based logic every 10 seconds.
    \item \textbf{Frontend Updates:} Pushes signal state via WebSocket.
    \item \textbf{Administrative APIs:} Serves endpoints for analytics, system status, and reporting.
\end{itemize}

\textbf{Example Endpoints:}
\begin{itemize}
    \item \texttt{POST /api/vehicle-count}
    \item \texttt{GET /api/signal-status}
    \item \texttt{WS /ws/live}
\end{itemize}

\section{Frontend Monitoring and Visualization}

The frontend application is implemented using Jetpack Compose Desktop. It provides:

\begin{itemize}
    \item \textbf{Dashboard:} Real-time status of each traffic light and vehicle density.
    \item \textbf{Live Feed Panel:} Optional integration of annotated snapshots.
    \item \textbf{Analytics View:} Graphs and charts showing wait times and flow rates.
    \item \textbf{Control Tools:} Manual override, system logs, and report export options.
\end{itemize}

State is managed using reactive streams (`StateFlow`) that listen for updates from the WebSocket channel.

\section{Deployment on Google Cloud}

The entire stack is deployed to a GCP virtual machine for 24/7 access and monitoring:

\begin{itemize}
    \item \textbf{VM Configuration:} Ubuntu 22.04 LTS, 4 vCPUs, 8 GB RAM.
    \item \textbf{Backend:} Ktor server daemonized via \texttt{systemd}.
    \item \textbf{AI Layer:} Python scripts initiated via supervisor or cron.
    \item \textbf{Security:} Only necessary ports exposed (e.g., 80/443), others firewalled.
    \item \textbf{Monitoring:} Logs are persisted and visualized via Prometheus and Grafana (optional).
\end{itemize}

\section{Evaluation Strategy}

The evaluation of the system includes both objective metrics and empirical observations:

\begin{itemize}
    \item \textbf{Detection Accuracy:} Verified against manually annotated ground truth \cite{jia2019traffic}.
    \item \textbf{Latency:} Measured from video frame capture to frontend update.
    \item \textbf{Signal Efficiency:} Compared average wait times against fixed-time baseline.
    \item \textbf{Robustness:} Tested under frame drop, camera failure, and traffic surge conditions.
\end{itemize}

Experiments were performed using real video samples from congested intersections in Dhaka (e.g., Mirpur 10).

\section{Summary}

This chapter has outlined the full technical methodology behind the AI-Based Traffic Management System, including data flow, architecture, algorithmic strategies, and deployment practices. By combining machine learning, computer vision, and reactive software design, the system is well-equipped to address real-world traffic challenges in developing cities \cite{zhao2017improving, chen2018ai}.
 \chapter{Data Presentation and Analysis}

This chapter presents the collected data from the AI-Based Traffic Management System and analyzes it to evaluate performance, detection accuracy, and traffic flow efficiency. The data was gathered from real-time or simulated video feeds at the Mirpur 10 intersection in Dhaka and includes vehicle counts, signal timings, and queue lengths. Visual tools such as tables and charts are used to enhance interpretability and derive meaningful insights from the raw data.

\section{Data Collection}

Data was gathered using the YOLOv8-powered object detection module\cite{yolov8}. A static overhead camera continuously captured video feeds, which were then processed using ONNX Runtime to detect and classify vehicles. For each 10-second cycle, the following information was recorded:

\begin{itemize}
    \item \textbf{Timestamp} — Time of capture for synchronization
    \item \textbf{Vehicle Count} — Per direction: North, South, East, and West
    \item \textbf{Vehicle Type} — Car, motorcycle, bus, and truck
    \item \textbf{Signal State} — Active green signal direction
    \item \textbf{Queue Estimation} — Based on object location and area coverage
\end{itemize}

Table~\ref{tab:vehicle_log} provides a sample of the raw data logged by the system.

\begin{table}[H]
\centering
\caption{Sample Vehicle Detection Log}
\label{tab:vehicle_log}
\begin{tabular}{|c|c|c|c|c|c|}
\hline
\textbf{Time} & \textbf{North} & \textbf{South} & \textbf{East} & \textbf{West} & \textbf{Green Signal} \\
\hline
10:00:00 & 15 & 8 & 12 & 5 & North \\
10:00:10 & 10 & 9 & 14 & 7 & East \\
10:00:20 & 6  & 15 & 8  & 10 & South \\
\hline
\end{tabular}
\end{table}

\section{Data Visualization}

To understand traffic behavior and signal responsiveness, several visualizations were generated.

\subsection{Vehicle Distribution by Direction}

\begin{figure}[H]
\centering
\includegraphics[width=0.9\textwidth]{Images/vehicle_distribution.png}
\caption{Total Vehicle Count per Direction over 10 Minutes}
\label{fig:vehicle_distribution}
\end{figure}

Figure~\ref{fig:vehicle_distribution} reveals consistently higher traffic density from the South and East approaches. This finding supports the adaptive signal prioritization logic used in the system~\cite{wang2020real, chen2018ai}.

\subsection{Signal Efficiency Comparison}

\begin{figure}[H]
\centering
\includegraphics[width=0.7\textwidth]{Images/signal_efficiency.png}
\caption{Average Waiting Time: Traditional vs AI-Based Signals}
\label{fig:signal_efficiency}
\end{figure}

As shown in Figure~\ref{fig:signal_efficiency}, the AI-based signal control system reduced average wait time by 30–40\%, especially during peak congestion.

\subsection{Vehicle Type Breakdown}

\begin{figure}[H]
\centering
\includegraphics[width=0.6\textwidth]{Images/vehicle_types.png}
\caption{Vehicle Types Detected (Sample Size: 1000 Vehicles)}
\label{fig:vehicle_types}
\end{figure}

In Figure~\ref{fig:vehicle_types}, private cars dominate the traffic composition, followed by motorcycles and buses. This breakdown aids in tuning the object detection model and resource allocation for each signal phase.

\section{Analysis and Insights}

The following insights emerged from the data:

\begin{itemize}
    \item \textbf{Directional Congestion:} Traffic is unevenly distributed, with South and East directions contributing to over 60\% of the total load.
    \item \textbf{Signal Responsiveness:} AI-based adaptive signaling reduced idle time per lane by up to 35\%.
    \item \textbf{Detection Accuracy:} Based on manual validation of sample frames, detection accuracy averaged 92\%~\cite{jia2019traffic}, with most false negatives occurring in occluded or overlapping vehicle scenarios.
    \item \textbf{Scalability:} The JSON-based output format allows for seamless expansion to multi-intersection environments~\cite{zhao2017improving} and integration with centralized traffic management dashboards.
\end{itemize}

\section{Limitations of the Data}

Despite promising results, the system faced a few limitations:

\begin{itemize}
    \item \textbf{Environmental Constraints:} Accuracy drops slightly during rain, extreme sunlight glare, and at night.
    \item \textbf{Camera Placement:} Detection performance varies based on elevation, tilt, and visibility of each lane.
    \item \textbf{Sample Scope:} The dataset is limited to one intersection and does not yet include long-term seasonal variations.
\end{itemize}

Future work should involve testing with multiple intersections, edge devices, and 24/7 deployments under varying weather conditions.

\section{Summary}

This chapter presented the collected data and analyzed the operational behavior of the AI-Based Traffic Management System. The results validate the system’s core capabilities in traffic optimization, with notable improvements in queue reduction, dynamic responsiveness, and detection precision. While the system shows clear potential, further iterations and broader deployments are needed to generalize these findings for large-scale urban applications.
 \chapter{Engineering Considerations}

The successful development and deployment of an AI-Based Traffic Management System require addressing multiple engineering factors. These considerations ensure that the system is not only functional but also reliable, maintainable, scalable, and secure in real-world conditions.

\section{System Architecture}

The system follows a modular architecture consisting of four main layers~\cite{modular_architecture}:

\begin{itemize}
    \item \textbf{Frontend Layer:} Built with Kotlin Jetpack Compose, it provides a responsive UI for dashboards, analytics, monitoring, reports, and live feed views.
    \item \textbf{Backend Layer:} Implemented using Ktor (Kotlin), it handles REST APIs, WebSocket-based real-time updates, and communication with the AI engine.
    \item \textbf{AI/ML Processing Layer:} Python-based modules using YOLOv8 perform vehicle detection, traffic analytics, and predictive modeling~\cite{yolov8}.
    \item \textbf{Infrastructure Layer:} Hosted on Google Cloud Platform using custom virtual machines for scalability, remote access, and load handling~\cite{gcp2023}.
\end{itemize}

\section{Hardware Requirements}

The core hardware requirements include:

\begin{itemize}
    \item High-resolution CCTV camera for real-time video capture.
    \item GPU-enabled virtual machine or edge device for YOLOv8 inference.
    \item Stable internet connection for continuous data transmission and remote access.
\end{itemize}

\section{Software Requirements}

\begin{itemize}
    \item \textbf{Frontend:} Android Studio, Jetpack Compose
    \item \textbf{Backend:} Kotlin + Ktor, PostgreSQL
    \item \textbf{AI:} Python 3.x, OpenCV, Ultralytics YOLOv8, NumPy
    \item \textbf{DevOps:} Docker, Git, Google Cloud SDK
\end{itemize}

\section{Scalability and Modularity}

The system is designed to be scalable by:

\begin{itemize}
    \item Supporting multiple intersections and cameras by deploying additional AI processing modules.
    \item Modular backend services allow horizontal scaling via containerization.
    \item Stateless backend architecture makes it easier to integrate with third-party systems like city dashboards.
\end{itemize}

\section{Security Considerations}

Security has been incorporated at every level~\cite{owasp}:

\begin{itemize}
    \item Secure API endpoints using JWT (JSON Web Tokens).
    \item HTTPS and SSL encryption for backend communication.
    \item Controlled access to video feeds and admin panels.
    \item Logging and monitoring of anomalies using server-side logs and audit trails.
\end{itemize}

\section{Reliability and Fault Tolerance}

To ensure high reliability:

\begin{itemize}
    \item AI detection fallback to last-known state in case of temporary failure.
    \item Redundant video buffer storage in case of connectivity loss.
    \item Heartbeat signals monitor system uptime, with auto-restart mechanisms.
\end{itemize}

\section{Sustainability and Maintenance}

\begin{itemize}
    \item Designed for long-term maintenance with separate configuration files and environment variables.
    \item Modular codebase with clean architecture ensures testability and ease of upgrades~\cite{clean_architecture}.
    \item Use of open-source libraries reduces cost and ensures community support.
\end{itemize}

\section{Legal and Ethical Considerations}

\begin{itemize}
    \item No facial recognition is performed, ensuring compliance with privacy laws~\cite{gdpr_ai}.
    \item System processes anonymized vehicle data only (e.g., type, count).
    \item Ensures public safety by not interfering with emergency signals.
\end{itemize}

\section{Summary}

This chapter outlined the core engineering decisions made during the design and implementation of the system. Each consideration—from architecture to privacy—was addressed to ensure the system is robust, secure, and scalable for future expansion in smart city infrastructure.
\end{lstlisting} 

Please note that you are free to rearrange the order of the chapters without considering their file name. 

\section{Adding Equations}
It is almost impossible to write a thesis or a technical report in an engineering discipline without equations. These equations contain mathematical functions, symbols, parameters, numbers, units, or any combination of them. In \LaTeX~adding equations is typeset-based and easy to use. For example typeset \verb|$\frac{a}{b}$| produces $\frac{a}{b}$, a simple typeset \verb|$\bar{x} = \frac{\sum_{i = 1}^n x_i}{n}$| produces an elegant inline equation $\bar{x} = \frac{\sum_{i = 1}^n x_i}{n}$, and a simple and easy looking typeset \verb|$\cos (2\theta) = \cos^2 \theta - \sin^2 \theta$| produces an inline equation $\cos (2\theta) = \cos^2 \theta - \sin^2 \theta$.


Numbered equations are used and often required in scientific writing. To produce a numbered equation, you need to use an \lstinline[language=latex]!equation! environment. The \lstinline[language=latex]!equation! environment automatically numbers your equation which can be referred from the document. A simple example of generating a numbered equation using \lstinline[language=latex]!equation! environment is as follows.
\lstset{language=latex, caption={}}
\begin{lstlisting}
\begin{equation} 
  x = \frac{-b \pm \sqrt{b^2 - 4ac}}{2a}
\end{equation}
\end{lstlisting}
The above set of commands adds the following numbered equation to your document.
\begin{tcolorbox}
\begin{equation} 
  x = \frac{-b \pm \sqrt{b^2 - 4ac}}{2a}
\end{equation}
\end{tcolorbox}
You can also use the \lstinline[language=latex]!\label! and \lstinline[language=latex]!\ref! commands to label and reference equations, respectively. The following equation shows an example of how to use \lstinline[language=latex]!\label! and \lstinline[language=latex]!\ref! commands in context.

\lstset{language=latex, caption={}}
\begin{lstlisting}
\begin{equation} \label{eq:2}
  E = mc^2
\end{equation}

The Eq. \ref{eq:2} defines the energy $E$ of a particle in its 
rest frame as the product of mass $m$ with the speed of light 
squared $c^2$.
\end{lstlisting}

The above piece of code produces the following output.
\begin{tcolorbox}
\begin{equation} \label{eq:2}
  E = mc^2
\end{equation}

The Eq. \ref{eq:2} defines the energy $E$ of a particle in its rest frame as the product of mass $m$ with the speed of light squared $c^2$. 
\end{tcolorbox}

\section{Adding a Section}

While writing a scientific document, it is necessary and often important to structure the content of the document into fundamental logic units. To achieve this, \LaTeX~ provides a command to generate section headings and number them automatically. The commands to create section headings are simple.
\lstset{language=latex, caption={}}
\begin{lstlisting}
\section{...} 
\end{lstlisting}

The \lstinline[language=latex]!\section{}! command is numbered and it appears in the table of contents of the document. 

\subsection{Adding a Subsection}
You can add a subsection like this one using \lstinline[language=latex]!\subsection{}! command and works in a similar manner as \lstinline[language=latex]!\section{}! does.

\subsubsection{Adding a Sub-subsection}
You can add a subsubsection like this one using \lstinline[language=latex]!\subsubsection{}! command. Sometimes you need to structure your report to a finer level and \lstinline[language=latex]!\subsubsection{}! the command helps you to achieve that intention.

\section{Adding a Figure}

``A picture is worth a thousand words'' --  a picture conveys information more effectively than words.
A picture has the power of telling a story. sometimes complex and multiple ideas can be depicted using an image that would otherwise be difficult to convey by a verbal description. The images, figures, graphs, and charts published within research articles are often the most important information in the paper. Photos can sum up key findings in a single image.

\begin{figure}[!ht]
    \centering
    \tikz{
      \node[anchor=south west] (image) at (0,0) {\includegraphics[width=0.8\textwidth]{Images/fig3.png}};
      \draw[GUBRed, line width=1mm] (2.3,0.9) ellipse (30pt and 10pt);
      \draw[GUBPurple, line width=1mm,-latex'] (7.7,5.2) -- (4.2,5.2) node [fill=black!5,above,pos=0.4,]{1. Click here};
      \draw[GUBPurple, line width=1mm,-latex'] (7.7,0.9) -- (3.4,0.9) node [fill=black!5,above,pos=0.23,]{2. Then click here};
    }
    \caption{Adding a new figure in the Images folder from the left side panel of the Overleaf of this template.}
    \label{fig:new_figure}
\end{figure}

There are many important reasons to use images for scientific and academic writing. However, most students do not realize it which results in losing the chance to make writing more eye-catching and interesting. This happens due to a lack of practice and lack of knowledge of academic writing. Additionally, sometimes students use images, graphics, and charts but they don’t know how to include pictures in a research paper correctly with proper citing.

In your report, if you want to add a figure properly then the first and foremost important task is to import the figure in the ``Images'' folder of this project. To do that, please follow the instructions given in Fig. \ref{fig:new_figure}. Then use the \textit{Add Files} dialogue box to complete uploading the image. When the figure is uploaded, you are now ready to use the image in your report.

Then add the figure to the desired location of your report using the following commands (assuming that the image you would like to add is Mars.jpg).
\lstset{language=latex, caption={}}
\begin{lstlisting}
\begin{figure}[ht]
    \centering
    \includegraphics[width=0.40\textwidth]{Images/Mars.jpg}
    \caption{Picture of the Planet Mars in natural color.}
    \label{fig:mars}
\end{figure}
\end{lstlisting}

In the above code, a figure is added having the caption ``Pictured of the Planet Mars in natural color.''. Additionally, you should add a \lstinline[language=latex]!\label{fig:mars}! to the figure by which you can refer the figure (e.g. \lstinline[language=latex]!\ref{fig:mars}!) from the body of the text. The command \lstinline[language=latex]!\centering! is used to center-align the figure.

You can also use the \lstinline[language=latex]!\label! and \lstinline[language=latex]!\ref! commands to label and reference figures, respectively. The following listing shows an example of how to use \lstinline[language=latex]!\label! and \lstinline[language=latex]!\ref! commands in context.

\lstset{language=latex, caption={}}
\begin{lstlisting}
\begin{figure}[ht]
  \centering
  \includegraphics[width=0.40\textwidth]{Images/Einstein.png}
  \caption{Einstein's official portrait after receiving the 
  1921 Nobel Prize in Physics.}
  \label{fig:einstein}
\end{figure}

In 1920, Albert Einstein became a Foreign Member of the Royal 
Netherlands Academy of Arts and Sciences. In 1922, he was 
awarded the 1921 Nobel Prize in Physics "for his services to 
Theoretical Physics, and especially for his discovery of the 
law of the photoelectric effect". Figure \ref{fig:einstein} 
shows his official portrait after receiving the prize. 
\end{lstlisting}

The above code produces the Figure \ref{fig:einstein} and the text below. Please have a careful look how \lstinline[language=latex]!\lebel{}! is used within \lstinline[language=latex]!figure! environment and how the figure is cited from the text using \lstinline[language=latex]!\ref{}! command.

\begin{figure}[ht]
  \centering
  \includegraphics[width=0.40\textwidth]{Images/Einstein.png}
  \caption{Einstein's official portrait after receiving the 
  1921 Nobel Prize in Physics.}
  \label{fig:einstein}
\end{figure}
\begin{tcolorbox}
In 1920, Albert Einstein became a Foreign Member of the Royal Netherlands Academy of Arts and Sciences. In 1922, he was awarded the 1921 Nobel Prize in Physics ``for his services to Theoretical Physics, and especially for his discovery of the law of the photoelectric effect''. Figure \ref{fig:einstein} shows his official portrait after receiving the prize. 
\end{tcolorbox}

\section{Adding a Table}
 Tables are an integral part of scientific writing. A table often used to summarize your findings clearly and neatly, and it allows the reader to understand the results and importance of the results. When you plan to include a table in your report, you should think carefully to present your data so that the readers understand it easily. Tables have several elements, including the caption, column titles, and body.
 \begin{itemize}
    \item Caption: Tables should have a clear caption with sufficient amount of detail to understand it standalone.
    \item Column Titles: Column titles are the headline of your data. A good set of column titles allow the reader to understand the table quickly.
    \item Body: This is the part of the table where data is presented.
 \end{itemize}

A table can be added in your documents by using a \lstinline[language=latex]!table! environment. The following code of a \lstinline[language=latex]!table! environment produces a table shown in Table \ref{tab:1}.
\lstset{language=latex, caption={}}
\begin{lstlisting}
\begin{table}[ht]
\centering
\caption{Data collected from four students.}
\begin{tabular}{l c c c}
\toprule
Name    & Weight (lb) & Height (in) & Gender \\ \toprule
Alice   & 133         & 65          & F      \\ \midrule
Bob     & 160         & 72          & M      \\ \midrule
Charlie & 152         & 70          & M      \\ \midrule
Diana   & 120         & 60          & F      \\ \bottomrule  
\end{tabular}
\label{tab:1}
\end{table}
\end{lstlisting}



\begin{table}[ht]
\centering
\caption{Data collected from four students.}
\begin{tabular}{l c c c}
\toprule
Name    & Weight (lb) & Height (in) & Gender \\ \toprule
Alice   & 133         & 65          & F      \\ \midrule
Bob     & 160         & 72          & M      \\ \midrule
Charlie & 152         & 70          & M      \\ \midrule
Diana   & 120         & 60          & F      \\ \bottomrule  
\end{tabular}
\label{tab:1}
\end{table}


\section{Citing Articles}

In academic writing, you need to read many scientific articles, reports or even authentic web documents to know the state of the art of your field of research. While preparing your report, you may need to use the information provided in those documents. If you consider doing so, you are allowed to do it giving credits to the original authors via citations. A citation is a reference to the source article of information that you used in your research. Any time you directly quote, paraphrase or summarize the essential elements of others idea in your work, you must cite the source. While directly quoting sentences from a published work, it should be surrounded by quotations marks with proper citation. Paraphrasing and summarizing or the published work are also allowed but you should cite the source properly. To cite an article, you may copy the BibTeX for the corresponding article from Google Scholar or any digital library. Please follow the steps provided in Figure \ref{fig:bibtex} to get the BibTeX of an article that you are looking for using Google Scholar.

\begin{figure}[ht]
  \centering
  \tikz{
      \node[anchor=south west] (image) at (0,0) {\includegraphics[width=0.9\textwidth]{Images/fig4.png}};
      \draw[GUBPurple, line width=1mm,-latex'] (4,6.5) -- (2,7.5) node [fill=black!5,above,pos=-0.2,]{1. https://scholar.google.com/};
      \draw[GUBPurple, line width=1mm,-latex'] (11,6.5) -- (9,7.5) node [fill=black!5,above,pos=-0.2,]{2. Search the article };
      \draw[GUBRed, line width=1mm] (4.2,4.6) ellipse (10pt and 10pt);
      \draw[GUBPurple, line width=1mm,-latex'] (8.4,4.6) -- (4.4,4.6) node [fill=black!5,above,pos=0.23,]{3. Click here};
      \draw[GUBPurple, line width=1mm,-latex'] (12.2,1.5) -- (8.2,1.5) node [fill=black!5,above,pos=0.23,]{4. Then click here};
    }
  \caption{BibTeX is readily available in Google Scholar.}
  \label{fig:bibtex}
\end{figure}


A typical BibTeX of an article looks as follows:
\lstset{language=latex, caption={}}
\begin{lstlisting}
@article{krizhevsky2012imagenet,
  title={Imagenet classification with deep convolutional 
  neural networks},
  author={Krizhevsky, Alex and Sutskever, Ilya and Hinton, 
  Geoffrey E},
  journal={Advances in neural information processing systems},
  volume={25},
  pages={1097--1105},
  year={2012}
}    
\end{lstlisting}

Get a required BibTeX and paste that in the \textit{references.bib} file of this template. Please refer to Figure \ref{fig:ref} to locate \textit{references.bib} file. You should take a note of the key to use in your report. In the above example \lstinline[language=latex]!krizhevsky2012imagenet! is the key.

\begin{figure}[ht]
  \centering
  \includegraphics[width=0.90\textwidth]{Images/fig5.png}
  \caption{Adding a new BibTeX entry in reference.bib from the left side panel of Overleaf of this template.}
  \label{fig:ref}
\end{figure}


Next, go to the desired location in your report to insert the reference. To cite this article, please write a command as follows: 
\lstset{language=latex, caption={}}
\begin{lstlisting}
Convolutional Neural Network revived with the success of 
\cite{krizhevsky2012imagenet} in ImageNet competition.
\end{lstlisting}

Next time you recompile, reference will be created as follows: 
\begin{tcolorbox}
Convolutional Neural Network revived with the success of \cite{krizhevsky2012imagenet} in ImageNet competition.
\end{tcolorbox}


Now, please go to the Bibliography section (you may jump to there by clicking on the number in green color as well) of your report where you will find bibliographic detail of your referred article.


% \subsection{Citing a Web Document}
% If you need to cite an authentic source of web information then you can fill up a BibTeX entry as follows.
% \lstset{language=latex, caption={}}
% \begin{lstlisting}
% @misc{seeta2011what,
%  title = {What is a liberal arts degree?},
%  author = {Seeta Bhardwa},
%  publisher = {Times Higher Education},
%  howpublished = {https://www.timeshighereducation.com/student/
%  advice/what-liberal-arts-degree},
%  year = {2020}
% }
% \end{lstlisting}
% In the following, the above BibTeX entry is cited in context. 
% \lstset{language=latex, caption={}}
% \begin{lstlisting}
% ``A liberal arts degree gives students the opportunity to 
% study many different areas of interest and to consider various 
% career options" \cite{seeta2011what}.
% \end{lstlisting}

% The above code produces the following output. You can check the BibTeX entry in the Bibliography section.
% \begin{tcolorbox}
% ``A liberal arts degree gives students the opportunity to study many different areas of interest and to consider various career options" \cite{seeta2011what}.
% \end{tcolorbox}



\section{Summary}
\LaTeX~allows you to produce your document with a great level of flexibility without compromising the quality of the output. This chapter has described how to use this template with two scenarios. However, this template is not restricted to use only for those two cases. The scenarios provided here only for illustration purpose and for better understanding of the students. You are allowed to use it in any combination of a group and the report type. For any support please send an email to muhammad.hasan@cse.green.edu.bd

