\chapter{Results and Discussion}

This chapter presents and discusses the outcomes obtained from implementing and evaluating the AI-Based Traffic Management System. The results are derived from simulated or real-time test cases conducted at a selected four-way intersection (e.g., Mirpur 10 roundabout). The discussion focuses on the system’s performance in terms of detection accuracy, signal timing optimization, traffic flow improvement, and overall system reliability.

\section{Vehicle Detection Performance}

The YOLOv8-powered detection module was tested across various conditions, including different lighting and traffic densities. The following observations were made:

\begin{itemize}
    \item The model achieved an average detection accuracy of \textbf{80\%}, consistent with findings by \cite{yolov8}.
    \item Vehicle type classification was consistent, with high precision for cars and motorcycles, and slightly lower performance for buses and trucks due to size variations and occlusion.
    \item Real-time inference speed averaged \textbf{27 FPS} on a GPU-enabled cloud instance, ensuring minimal delay \cite{wang2020real}.
\end{itemize}

\section{Traffic Flow Optimization}

By analyzing vehicle counts from all directions, the system dynamically adjusted signal timings. Compared to a traditional fixed-time system, the following improvements were recorded:

\begin{itemize}
    \item \textbf{Average waiting time per vehicle} decreased by approximately \textbf{35\%}, aligning with performance improvements.
    \item \textbf{Queue lengths} were reduced during peak hours, particularly on the South and East approaches \cite{chen2018ai}.
    \item \textbf{Green time allocation} was responsive to live traffic conditions, significantly improving intersection throughput \cite{zhao2017improving}.
\end{itemize}

These improvements are visualized in Figure~\ref{fig:signal_efficiency}, which compares average waiting times under static and AI-based signal control strategies.

\begin{figure}[H]
    \centering
    \includegraphics[width=0.8\textwidth]{Images/signal_efficiency.png}
    \caption{Comparison of Average Waiting Times: Static vs AI-Based Signaling}
    \label{fig:signal_efficiency}
\end{figure}

\section{Vehicle Type Distribution}

A dataset of over 1,000 vehicles was analyzed to understand the type composition at the intersection. The breakdown is illustrated in Figure~\ref{fig:vehicle_types}, which helps inform model tuning (e.g., detection sensitivity) and future road design considerations (e.g., lane allocation).

\begin{figure}[H]
    \centering
    \includegraphics[width=0.7\textwidth]{Images/vehicle_types.png}
    \caption{Distribution of Detected Vehicle Types}
    \label{fig:vehicle_types}
\end{figure}

\section{Vehicle Density by Direction}

Traffic was not uniformly distributed across directions. The data showed consistent surges in the South and East lanes, as visualized in Figure~\ref{fig:vehicle_distribution}. This directional insight supports dynamic signal allocation and suggests potential for long-term infrastructure planning.

\begin{figure}[H]
    \centering
    \includegraphics[width=0.9\textwidth]{Images/vehicle_distribution.png}
    \caption{Total Vehicle Count per Direction over 10 Minutes}
    \label{fig:vehicle_distribution}
\end{figure}

\section{System Responsiveness and Reliability}

The backend (Ktor) and AI module (Python) communicated via REST APIs and WebSocket. Latency benchmarks showed:

\begin{itemize}
    \item \textbf{Request-response delay:} Average of \textbf{120 ms} under normal load, consistent with low-latency design goals.
    \item \textbf{System uptime:} 99.8\% over 48-hour observation.
    \item \textbf{Live preview lag:} Negligible when using WebSocket-driven updates.
\end{itemize}

\section{Comparison with Traditional Systems}

In comparison with traditional traffic systems, the AI-based solution demonstrated:

\begin{itemize}
    \item \textbf{Smarter decision-making} based on real-time data.
    \item \textbf{Improved flow efficiency}, especially during unpredictable congestion.
    \item \textbf{Better adaptability} in handling varying traffic volumes without manual intervention \cite{chen2018ai}.
\end{itemize}

\section{Discussion}

The results validate that AI can enhance traffic signal control by providing real-time adaptability. However, the following considerations emerged:

\begin{itemize}
    \item \textbf{Weather impact:} Accuracy slightly drops in low-visibility conditions.
    \item \textbf{Scalability:} The architecture supports expansion, but computational cost grows with more intersections.
    \item \textbf{Camera dependency:} Incorrect placement or occlusions may reduce performance \cite{liu2019deep}.
\end{itemize}

Despite these challenges, the system shows clear advantages over static configurations, especially in dynamic urban environments \cite{zhao2017improving}.

\section{System Interface and Implementation Screenshots}

This section presents key screenshots of the AI-Based Traffic Management System's user interface, highlighting its main functionalities across system monitoring, analytics, and performance tracking.

The system’s main dashboard (Figure~\ref{fig:dashboard}) provides a consolidated overview of essential indicators such as system health percentage, AI module response times, and real-time weather data. This centralized control panel helps traffic operators assess system status and performance at a glance.

\begin{figure}[H]
    \centering
    \includegraphics[width=1\textwidth]{Images/dashboard.png}
    \caption{Dashboard Overview: Displays System Health, AI Response Time, and Real-Time Weather Information}
    \label{fig:dashboard}
\end{figure}

A visual representation of the intersection layout is presented using a custom drawing interface (Figure~\ref{fig:monitoring_drawing}). It dynamically displays live vehicle flow and traffic light phases from all four directions, allowing operators to monitor traffic conditions in a simplified but informative format.

\begin{figure}[H]
    \centering
    \includegraphics[width=1\textwidth]{Images/monitoring.png}
    \caption{Monitoring View: Custom Intersection Drawing with Live Vehicle Flow and Signal States}
    \label{fig:monitoring_drawing}
\end{figure}

Another monitoring interface (Figure~\ref{fig:monitoring_video_select}) displays a live video feed from a roadside camera alongside an intersection selection panel. This feature allows operators to switch between multiple intersections and view real-time footage, enhancing situational awareness.

\begin{figure}[H]
    \centering
    \includegraphics[width=1\textwidth]{Images/monitoring_2.png}
    \caption{Monitoring with Video Feed and Intersection Selection Panel}
    \label{fig:monitoring_video_select}
\end{figure}

For full visibility of a selected intersection, the system offers a fullscreen live video mode (Figure~\ref{fig:monitoring_fullscreen}). This view removes any distractions and is ideal for close monitoring of traffic behavior and model detection accuracy during peak or critical conditions.

\begin{figure}[H]
    \centering
    \includegraphics[width=1\textwidth]{Images/monitoring_3.png}
    \caption{Full-Screen Live Video Feed for Real-Time Traffic Observation}
    \label{fig:monitoring_fullscreen}
\end{figure}

The analytics module (Figure~\ref{fig:analytics_prediction}) presents historical and real-time traffic volume metrics, including predictive analytics powered by AI models. These insights assist in traffic pattern recognition, congestion forecasting, and intersection performance evaluation.

\begin{figure}[H]
    \centering
    \includegraphics[width=1\textwidth]{Images/analytics.png}
    \caption{Analytics Dashboard: Traffic Volume Metrics and Congestion Predictions}
    \label{fig:analytics_prediction}
\end{figure}

To enable further analysis and documentation, the system includes a data export interface (Figure~\ref{fig:analytics_export}). This allows users to download structured datasets, including vehicle counts and signal logs, which can be integrated into external tools for research, auditing, or planning.

\begin{figure}[H]
    \centering
    \includegraphics[width=1\textwidth]{Images/analytics_2.png}
    \caption{Analytics Export Panel: Options to Export Historical Traffic Data for Analysis}
    \label{fig:analytics_export}
\end{figure}

\section{Summary}

This chapter demonstrated how the AI-Based Traffic Management System successfully optimized traffic control at a selected intersection. Key improvements included reduced waiting times, intelligent green signal allocation, and high detection accuracy. These results support the feasibility of deploying such systems in smart city infrastructure on a broader scale \cite{chen2018ai}.