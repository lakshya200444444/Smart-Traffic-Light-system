\chapter{Data Presentation and Analysis}

This chapter presents the collected data from the AI-Based Traffic Management System and analyzes it to evaluate performance, detection accuracy, and traffic flow efficiency. The data was gathered from real-time or simulated video feeds at the Mirpur 10 intersection in Dhaka and includes vehicle counts, signal timings, and queue lengths. Visual tools such as tables and charts are used to enhance interpretability and derive meaningful insights from the raw data.

\section{Data Collection}

Data was gathered using the YOLOv8-powered object detection module\cite{yolov8}. A static overhead camera continuously captured video feeds, which were then processed using ONNX Runtime to detect and classify vehicles. For each 10-second cycle, the following information was recorded:

\begin{itemize}
    \item \textbf{Timestamp} — Time of capture for synchronization
    \item \textbf{Vehicle Count} — Per direction: North, South, East, and West
    \item \textbf{Vehicle Type} — Car, motorcycle, bus, and truck
    \item \textbf{Signal State} — Active green signal direction
    \item \textbf{Queue Estimation} — Based on object location and area coverage
\end{itemize}

Table~\ref{tab:vehicle_log} provides a sample of the raw data logged by the system.

\begin{table}[H]
\centering
\caption{Sample Vehicle Detection Log}
\label{tab:vehicle_log}
\begin{tabular}{|c|c|c|c|c|c|}
\hline
\textbf{Time} & \textbf{North} & \textbf{South} & \textbf{East} & \textbf{West} & \textbf{Green Signal} \\
\hline
10:00:00 & 15 & 8 & 12 & 5 & North \\
10:00:10 & 10 & 9 & 14 & 7 & East \\
10:00:20 & 6  & 15 & 8  & 10 & South \\
\hline
\end{tabular}
\end{table}

\section{Data Visualization}

To understand traffic behavior and signal responsiveness, several visualizations were generated.

\subsection{Vehicle Distribution by Direction}

\begin{figure}[H]
\centering
\includegraphics[width=0.9\textwidth]{Images/vehicle_distribution.png}
\caption{Total Vehicle Count per Direction over 10 Minutes}
\label{fig:vehicle_distribution}
\end{figure}

Figure~\ref{fig:vehicle_distribution} reveals consistently higher traffic density from the South and East approaches. This finding supports the adaptive signal prioritization logic used in the system~\cite{wang2020real, chen2018ai}.

\subsection{Signal Efficiency Comparison}

\begin{figure}[H]
\centering
\includegraphics[width=0.7\textwidth]{Images/signal_efficiency.png}
\caption{Average Waiting Time: Traditional vs AI-Based Signals}
\label{fig:signal_efficiency}
\end{figure}

As shown in Figure~\ref{fig:signal_efficiency}, the AI-based signal control system reduced average wait time by 30–40\%, especially during peak congestion.

\subsection{Vehicle Type Breakdown}

\begin{figure}[H]
\centering
\includegraphics[width=0.6\textwidth]{Images/vehicle_types.png}
\caption{Vehicle Types Detected (Sample Size: 1000 Vehicles)}
\label{fig:vehicle_types}
\end{figure}

In Figure~\ref{fig:vehicle_types}, private cars dominate the traffic composition, followed by motorcycles and buses. This breakdown aids in tuning the object detection model and resource allocation for each signal phase.

\section{Analysis and Insights}

The following insights emerged from the data:

\begin{itemize}
    \item \textbf{Directional Congestion:} Traffic is unevenly distributed, with South and East directions contributing to over 60\% of the total load.
    \item \textbf{Signal Responsiveness:} AI-based adaptive signaling reduced idle time per lane by up to 35\%.
    \item \textbf{Detection Accuracy:} Based on manual validation of sample frames, detection accuracy averaged 92\%~\cite{jia2019traffic}, with most false negatives occurring in occluded or overlapping vehicle scenarios.
    \item \textbf{Scalability:} The JSON-based output format allows for seamless expansion to multi-intersection environments~\cite{zhao2017improving} and integration with centralized traffic management dashboards.
\end{itemize}

\section{Limitations of the Data}

Despite promising results, the system faced a few limitations:

\begin{itemize}
    \item \textbf{Environmental Constraints:} Accuracy drops slightly during rain, extreme sunlight glare, and at night.
    \item \textbf{Camera Placement:} Detection performance varies based on elevation, tilt, and visibility of each lane.
    \item \textbf{Sample Scope:} The dataset is limited to one intersection and does not yet include long-term seasonal variations.
\end{itemize}

Future work should involve testing with multiple intersections, edge devices, and 24/7 deployments under varying weather conditions.

\section{Summary}

This chapter presented the collected data and analyzed the operational behavior of the AI-Based Traffic Management System. The results validate the system’s core capabilities in traffic optimization, with notable improvements in queue reduction, dynamic responsiveness, and detection precision. While the system shows clear potential, further iterations and broader deployments are needed to generalize these findings for large-scale urban applications.