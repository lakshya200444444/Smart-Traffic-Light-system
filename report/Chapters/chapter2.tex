\chapter{Literature Review}

Effective traffic management has become a growing challenge in modern urban areas due to the exponential increase in the number of vehicles and limited road infrastructure. Traditional traffic control systems, which operate based on pre-programmed signal timings, often fail to adapt to dynamic and unpredictable traffic conditions. This results in prolonged waiting times, fuel wastage, and increased levels of air pollution \cite{zhao2017improving}. To overcome these limitations, researchers have increasingly turned to Artificial Intelligence (AI) technologies as a means to enhance the efficiency and responsiveness of traffic management systems.

\section{Review of Existing Literature}

Numerous studies in recent years have demonstrated the potential of AI in optimizing traffic flow. One of the most widely explored techniques is \textbf{machine learning}, particularly \textbf{reinforcement learning}, which enables traffic signal controllers to learn optimal timing policies based on real-time data. These algorithms are capable of minimizing average waiting times and vehicle congestion by continuously adapting to changing traffic patterns. For instance, research by Wang et al. has shown that Deep Q-Network (DQN)-based approaches can outperform traditional fixed-cycle signals by dynamically adjusting green light durations \cite{wang2020real}.

Another significant area of research is \textbf{computer vision}, where techniques such as \textbf{object detection} and \textbf{image classification} are applied to live video feeds from CCTV cameras. Using Convolutional Neural Networks (CNNs), systems can detect and count vehicles at intersections, enabling real-time estimation of traffic density. This method eliminates the need for expensive infrastructure like inductive loop detectors and provides a scalable solution for urban monitoring \cite{chen2018ai}.

In addition, the integration of \textbf{Internet of Things (IoT)} devices has allowed traffic management systems to gather diverse types of data, including vehicle speed, density, and road occupancy. These IoT-based systems use sensors and GPS-enabled units to feed data into AI models, which then make informed decisions about signal timing, congestion prediction, and emergency vehicle prioritization. Several smart cities have started implementing such systems on a trial basis with promising results \cite{jia2019traffic}.

Projects like \textit{Surtrac}, developed by Carnegie Mellon University, and \textit{SCATS} (Sydney Coordinated Adaptive Traffic System) have demonstrated the real-world applicability of AI in traffic control. These systems utilize decentralized decision-making and real-time data analysis to significantly reduce travel time and vehicle idling \cite{zhao2017improving}. However, most of these systems are implemented in technologically advanced urban environments with high infrastructure budgets.

\section{Knowledge Gaps and Research Motivation}

Despite significant progress, many existing AI-based traffic systems are still limited in scope and implementation. Key challenges include the need for high-quality, real-time data, integration with legacy infrastructure, and the computational complexity of AI models. Moreover, many systems are designed for high-income urban areas and do not account for traffic conditions in developing countries, where infrastructure and resource constraints are more pronounced.

This research aims to address these gaps by designing and developing a real-time AI-based traffic management system that uses computer vision for vehicle detection and a machine learning model to dynamically adjust traffic signals. The system is intended to be low-cost, scalable, and suitable for deployment in congested intersections of developing cities. By building upon the foundations laid in previous research and addressing the identified limitations, this study contributes a practical, efficient, and adaptable solution for modern traffic management challenges \cite{yolov8}.